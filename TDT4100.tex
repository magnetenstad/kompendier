\documentclass{article}

% Language setting
% Replace `english' with e.g. `spanish' to change the document language
%\usepackage[english]{babel}

% Set page size and margins
% Replace `letterpaper' with`a4paper' for UK/EU standard size
\usepackage[letterpaper,top=2cm,bottom=2cm,left=3cm,right=3cm,marginparwidth=1.75cm]{geometry}

% Useful packages
\usepackage{amsmath}
\usepackage{graphicx}
\usepackage[colorlinks=true, allcolors=blue]{hyperref}
\usepackage{tikz}
\usepackage{minted}
\usepackage{color}
\usepackage{courier}
\usemintedstyle{borland}

% Number sets
\usepackage{amsfonts}

\title{Objektorientert programmering}
\author{Magne Tenstad \\ Karin Syversveen Lie}

\begin{document}


\maketitle

\clearpage

\tableofcontents

\clearpage


\section{Huskeliste}
Etter å ha skrevet en klasse er det nyttig å sjekke at man har:
\begin{enumerate}
    \item Arvet og/eller implementert riktige superklasser/grensesnitt.
    \item Valgt riktig av class, abstract class og interface.
    \item Gjort felter private (eller protected ved arv).
    \item Gjort felter som ikke skal endres final.
    \item Initialisert felter som ikke skal være null, enten ved deklarasjon eller i konstruktør.
    \item For hver metode:
    \begin{enumerate}
        \item Gjort private dersom den bare brukes internt, public ellers.
        \item Validert argumentene.
        \item Kastet hensiktsmessige exceptions.
        \item Innkapsling: returnere \textit{kopi av} f.eks. liste
        \item Gjort static dersom den er helt uavhengig av objektets tilstand.
        \item Skrevet @Override dersom den overskriver.
        \item Gjort abstract dersom den ikke er implementert og klassen er abstract.
    \end{enumerate}
    \item Fikset alle error (rød strek).
    \item Testet metodene vha. en main-metode.
    \item Importert riktig og fjernet ubrukte importeringer.
\end{enumerate}

\section{Java Grensesnitt}

\subsection{Collection og List}


\subsection{Iterator og Iterable}

\subsubsection{Metoder}
Iterator-grensesnittet har metodene
\begin{itemize}
    \item boolean hasNext();
    \item T next();
\end{itemize}
Iterable-grensesnittet har metoden
\begin{itemize}
    \item Iterator$<$T$>$ iterator();
\end{itemize}
Man kan altså alltid får en Iterator fra en Iterable.

\subsubsection{Iterasjon}

Å iterere gjennom en iterator
\begin{minted}[]{java}
Iterator<T> iterator;
while (iterator.hasNext()) {
    T element = iterator.next();
}
\end{minted}
Å iterere gjennom en iterable
\begin{minted}[]{java}
Iterable<T> iterable;
for (T element : iterable) {

}
\end{minted}
2
\section{Java Funksjonelle Grensesnitt}

\subsection{Predicate}
// TODO

\subsection{Comparator og Comparable}
Klasser som implementerer Comparator-grensesnittet har en metode int compare().\\Eksempel:
\begin{minted}[]{java}
public int compare(Person o1, Person o2){
    return o1.getAge()- o2.getAge()
}
\end{minted}
Man kan også definere en Comparator som en funksjon:
\begin{minted}[]{java}
Comparator<Book> comparator = (book1, book2) ->
    book1.getName().compareTo(book2.getName());
\end{minted}
Klasser som implementerer Comparable har en metode compareTo(). En Collection med objekter som er Comparable kan sorteres ved Collections.sort(collection). String og Integer er eksempler på klasser som er Comparable. 


\section{Java Klasser}

\subsection{Scanner}
\subsubsection{Lese fra fil}
\begin{minted}[]{java}
File file = new File(filename);
try {
    Scanner scanner = new Scanner(file);
    while(scanner.hasNextLine()) {
        String line = scanner.nextLine();
        // Do something with the line
    }
    scanner.close();
} catch (FileNotFoundException e) {
    e.printStackTrace();
}
\end{minted}

\subsubsection{Lese fra stream (og user input)}
\begin{minted}[]{java}
Stream stream = System.in; // For user input
Scanner scanner = new Scanner(stream);
while(scanner.hasNextLine()) {
    String line = scanner.nextLine();
    // Do something with the line
}
\end{minted}

\subsection{String}
\subsubsection{Strengmanipulasjon}
Man kan dele opp strenger ved bruk av s.substring(pos).

\begin{itemize}
    \item String s.substring(pos)
    \item boolean s.endsWith("]")
    \item boolean s.startsWith("[")
    \item String[] s.split(",")
\end{itemize}

Skal man samle sammen strenger bruk StringBuilder. Her kan man ta builder.append(s)


\subsection{Random}
\subsubsection{Math.random()}
Math.random() returnerer en tilfeldig double mellom 0 og 1.

\subsubsection{Random double $\in [min,\,max]$}
\begin{minted}[]{Java}
public static double random_range(double min, double max) {
    return Math.random() * (max - min + 1) + min;
}
\end{minted}
\subsubsection{Random int $\in [min,\,max]$}
\begin{minted}[]{Java}
public static int random_range(int min, int max) {
    return (int) (Math.random() * (max - min + 1)) + min;
}
\end{minted}



\subsection{Stream}
Ved hjelp av streams kan man forenkle en rekke problemer som har å gjøre med sett av objekter.\\
Stream fra Collection:
\begin{minted}[]{Java}
    Collection<Object> objects;
    objects.stream()...
\end{minted}
Stream fra array:
\begin{minted}[]{Java}
    Object[] objects;
    Arrays.stream(objects)...
\end{minted}

\subsubsection{collect - fra liste til stream og tilbake}
\begin{minted}[]{java}
books = books.stream()
    .collect(Collectors.toList());
\end{minted}


\subsubsection{filter - bestemme et kriterium}
\begin{minted}[]{java}
List<Book> longBooks;
longBooks = books.stream()
    .filter(book -> book.getPages() > 500)
    .collect(Collectors.toList());
\end{minted}


\subsubsection{findAny - hente ut et element}
\begin{minted}[]{java}
Book myBook;
Person me;
myBook = books.stream()
    .filter(book -> book.getOwner() == me)
    .findAny()
    .orElse(null);
\end{minted}


\subsubsection{count - telle antall i streamen}
\begin{minted}[]{java}
int numberOfBooksIOwn;
Person me;
numberOfBooksIOwn = (int) books.stream()
    .filter(book -> book.getOwner() == me)
    .count();
\end{minted}


\subsubsection{sorted - sortere en stream}
\begin{minted}[]{java}
Comparator<Book> comparator = (book1, book2) -> {
    return book1.getName().compareTo(book2.getName());
};
List<Book> sortedBooks;
sortedBooks = books.stream()
    .sorted(comparator)
    .collect(Collectors.toList());
\end{minted}


\subsubsection{anyMatch}
\begin{minted}[]{java}
boolean doesHarryPotterExist;
doesHarryPotterExist = books.stream()
    .anyMatch(book -> book.getName().equals("Harry Potter"));
\end{minted}

\section{Teknikker}

\subsection{Observeratør og observert}
Denne teknikken sørger for at informasjon oppdateres i nødvendige objekter slik at den er konsistent. Man lager gjerne en interface for observatøren. Merk: \textit{Observer} og \textit{listener} omtaler det samme.\\\\
Nyttige metoder i observert:
\begin{itemize}
    \item addListener
    \item removeListener
    \item notifyListeners
\end{itemize}
Nyttige metoder i observatør:
\begin{itemize}
    \item receiveUpdate
\end{itemize}


\section{JUnit 5 - Tester}
Importering av junit skal gjerne være på formen:
\begin{minted}[]{java}
import static org.junit.jupiter.api.Assertions.*;
import org.junit.jupiter.api.*;
\end{minted}

\subsection{Metoder}
\begin{itemize}
    \item assertTrue
    \item assertFalse
    \item assertEquals
    \item assertNotEquals
    \item assertThrows
    \item assertNull
    \item fail
\end{itemize}

\subsection{Annotasjoner}
\begin{itemize}
    \item @BeforeAll
    \item @BeforeEach
    \item @Test
    \item @AfterEach
    \item @AfterAll
\end{itemize}

\section{FXML}
\subsection{Gettere og settere}
For de fleste klasser i JavaFX gjelder det at en henter verdi/tekst ved .getValue()/.getText() og setter verdi/tekst ved .setValue()/.setText().

\subsection{Rullegardin}
Hvis det er flere valgmuligheter så må man bruke en metode for å hente den

Eksempel
\begin{minted}[]{java}
char item = typeSelector.getSelectionModel().getSelectedItem();
\end{minted}



\end{document}

