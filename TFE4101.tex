\documentclass{article}

% Language setting
% Replace `english' with e.g. `spanish' to change the document language
%\usepackage[english]{babel}

% Set page size and margins
% Replace `letterpaper' with`a4paper' for UK/EU standard size
\usepackage[letterpaper,top=2cm,bottom=2cm,left=3cm,right=3cm,marginparwidth=1.75cm]{geometry}

% Useful packages
\usepackage{amsmath}
\usepackage{graphicx}
\usepackage[colorlinks=true, allcolors=blue]{hyperref}

% Number sets
\usepackage{amsfonts}

\title{Krets og digitalteknikk}
\author{Magne Tenstad}

\begin{document}

\maketitle

\clearpage

\tableofcontents

\clearpage

\section{Krets}


\subsection{Terminologi og definisjoner}


\subsubsection{Terminologi}
\begin{table}[h]
    \centering
    \begin{tabular}{c|c|c}
                    & Symbol & Enhet \\
        \hline
        Ladning     & Q & C, coulomb \\
        Strøm       & I & A, ampere \\
        Resistans   & R & $\Omega$, ohm \\
        Spenning    & U & V, volt \\
        Energi      & W & J, joule \\
        Effekt      & P & W, watt \\
        Konduktans  & G & S, siemens \\
        Fluks       & $\Phi$ & Wb, weber \\
        Induktans   & L & H, henry \\
        Kapitans    & C & F, farad \\
        Feltstyrke  & $\epsilon$ & V/m
    \end{tabular}
\end{table}


\subsubsection{Passiv fortegnskonvensjon}
Gjennom en kilde går strømretningen inn på negativ terminal og ut fra positiv terminal. Gjennom en motstand går strømretningen inn på positiv terminal og ut fra negativ terminal.


\subsubsection{Ohms lov}
Proposjonalitet mellom spenning over og strøm gjennom en motstand.
\[U = R \cdot I\]
med $P = U \cdot I$ følger det at
\[P = R \cdot I^2\]
og at
\[P = \frac{U^2}{R}\]


\subsubsection{Konduktans}
Konduktans angir ledningsevne, og er gitt ved
\[G = \frac{1}{R}\]
Konduktans kan være praktisk der resistansen er og skal være lav.


\subsubsection{Effekt og passiv fortegnskonvensjon}
La $k$ være en komponent, $U$ være spenningen over $k$ og $I$ være strømmen gjennom $k$. Effekten forbrukt i komponenten er gitt ved
\begin{itemize}
    \item Hvis strømretningen oppfyller den passive fortegnskonvensjonen: $P = U \cdot I$
    \item Hvis strømretningen ikke oppfyller den passive fortegnskonvensjonen: $P = - U \cdot I$
\end{itemize}

\subsubsection{Resistans}
Total resistans i seriekobling er gitt ved
\begin{align*}
    R_{ser} &= \sum_i R_i \\
    &= R_1 + R_2 + ... + R_n
\end{align*}
Total resistans i parallellkobling er gitt ved
\begin{align*}
    R_{par} &= \left( \sum_i \frac{1}{R_i} \right)^{-1} \\
    &= \left(\frac{1}{R_1} + \frac{1}{R_2} + ... + \frac{1}{R_n}\right)^{-1}
\end{align*}
Dersom kretsen ikke kan beskrives ved en kombinasjon av serie og parallellkoblinger, må det gjøres en tilnærming. Eks: H20 Oppgave 1.


\subsubsection{Strøm}
I en seriekobling er strømmen konstant.\\\\
La $I$ være den totale strømmen gjennom en parallellkobling, $R$ være summen av resistansene i koblingen og $R'$ være resistansen i en gren. Da er strømmen gjennom grenen gitt ved
\[I' = I \, \frac{R - R'}{R}\]


\subsection{Spole}
Ved $t = 0$ er det kontinuitet i spolestrømmen. Ved $t \rightarrow \infty \implies u(t) \rightarrow 0$\\\\
Fluksen gjennom en spole er gitt ved
\[\Phi = L \cdot I\]
Spenning
\[u(t) = L \cdot i\,'(t)\]

\subsection{Kondensator}
Spenningsdifferansen i en kondensator produserer et elektrisk felt.
Feltstyrken er gitt ved
\[\epsilon = \frac{U}{d}\]
der $d$ er avstanden.\\\\
Ved $t = 0$ er det kontinuitet i kondensatorspenning. Ved $t \rightarrow \infty \implies i(t) \rightarrow 0$\\\\
Strøm
\[i(t) = C \cdot u\,'(t)\]
Spenning
\[u(t) = U_0+\frac{1}{C}\,\int_0^t i(x)\,dx\]

\subsubsection{Kapitans}
Er gitt ved
\[C = \frac{Q}{U}\]
Total kapitans i seriekobling er gitt ved
\begin{align*}
    C_{ser} &= \left( \sum_i \frac{1}{C_i} \right)^{-1} \\
    &= \left(\frac{1}{C_1} + \frac{1}{C_2} + ... + \frac{1}{C_n}\right)^{-1}
\end{align*}
Total kapitans i parallellkobling er gitt ved
\begin{align*}
    C_{par} &= \sum_i C_i \\
    &= C_1 + C_2 + ... + C_n
\end{align*}


\subsection{Superposisjonsmetoden}
Kan brukes for lineære mål som strøm og spenning, men ikke kvadratiske mål som effekt.







\section{Digitalteknikk}





\end{document}