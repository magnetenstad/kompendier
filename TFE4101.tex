\documentclass{article}

% Language setting
% Replace `english' with e.g. `spanish' to change the document language
%\usepackage[english]{babel}

% Set page size and margins
% Replace `letterpaper' with`a4paper' for UK/EU standard size
\usepackage[letterpaper,top=2cm,bottom=2cm,left=3cm,right=3cm,marginparwidth=1.75cm]{geometry}

% Useful packages
\usepackage{amsmath}
\usepackage{graphicx}
\usepackage[colorlinks=true, allcolors=blue]{hyperref}

% Number sets
\usepackage{amsfonts}

\title{Krets og digitalteknikk}
\author{Magne Tenstad}

\begin{document}

\maketitle

\clearpage

\tableofcontents

\clearpage

\section{Krets}


\subsection{Terminologi og definisjoner}


\subsubsection{Symboler og enheter}
\begin{table}[h]
    \centering
    \begin{tabular}{c|c|c}
                    & Symbol & Enhet \\
        \hline
        Ladning     & Q & C, coulomb \\
        Strøm       & I & A, ampere \\
        Resistans   & R & $\Omega$, ohm \\
        Spenning    & U & V, volt \\
        Energi      & W & J, joule \\
        Effekt      & P & W, watt \\
        Konduktans  & G & S, siemens \\
        Fluks       & $\Phi$ & Wb, weber \\
        Induktans   & L & H, henry \\
        Kapitans    & C & F, farad \\
        Feltstyrke  & $\epsilon$ & V/m
    \end{tabular}
\end{table}

\subsubsection{Ulike typer komponenter}
\begin{table}[h]
    \centering
    \begin{tabular}{c|c|c}
    Komponent       & Aktiv / Passiv    & Lineær / Ulineær \\
    \hline
    Spenningskilder & -         & Lineær\\
    Strømkilder     & -         & Lineær\\
    Resistanser     & Passiv    & Lineær\\
    Spoler          & Passiv    & Lineær\\
    Kondensatorer   & Passiv    & Lineær\\
    Dioder          & Aktiv     & Ulineær\\
    Transistorer    & Aktiv     & Ulineær
    \end{tabular}
\end{table}

\subsubsection{Passiv fortegnskonvensjon}
Når referanseretningen for strømmen gjennom et kretselement er i samme retning
som referansespenningsfallet over elementet (inn på positiv terminal, ut på negativ), skal positivt fortegn brukes i uttrykk som relaterer spenningen til strømmen. Ellers skal negativt fortegn brukes.

\subsubsection{Ohms lov}
Proposjonalitet mellom spenning over og strøm gjennom en motstand.
\[U = R \cdot I\]
med $P = U \cdot I$ følger det at
\[P = R \cdot I^2\]
og at
\[P = \frac{U^2}{R}\]


\subsubsection{Konduktans}
Konduktans angir ledningsevne, og er gitt ved
\[G = \frac{1}{R}\]
Konduktans kan være praktisk der resistansen er og skal være lav.


\subsubsection{Resistans}
Total resistans i seriekobling er gitt ved
\begin{align*}
    R_{ser} &= \sum_i R_i \\
    &= R_1 + R_2 + ... + R_n
\end{align*}
Total resistans i parallellkobling er gitt ved
\begin{align*}
    R_{par} &= \left( \sum_i \frac{1}{R_i} \right)^{-1} \\
    &= \left(\frac{1}{R_1} + \frac{1}{R_2} + ... + \frac{1}{R_n}\right)^{-1}
\end{align*}
Dersom kretsen ikke kan beskrives ved en kombinasjon av serie og parallellkoblinger, må det gjøres en tilnærming. Eks: H20 Oppgave 1.


\subsection{Strømdeling}
I en seriekobling er strømmen konstant.\\\\
La $I$ være den totale strømmen gjennom en parallellkobling, $R_{par}$ være den totale resistansen i koblingen og $R'$ være resistansen i en gren. Da er strømmen gjennom grenen gitt ved
\[I' = I \, \frac{R_{par}}{R'}\]

\subsection{Spenningsdeling}
I en parallellkobling er spenningen konstant.\\\\
La $U$ være den totale strømmen gjennom en seriekobling, $R_{ser}$ være den totale resistansen i koblingen og $R'$ være en resistansen til en komponent. Da er spenningen over komponenten gitt ved
\[U' = U \, \frac{R'}{R_{ser}}\]

\subsection{Spole}
Ved $t = 0$ er det kontinuitet i spolestrømmen. Ved $t \rightarrow \infty \implies u(t) \rightarrow 0$ (kortslutning)\\\\
Fluksen gjennom en spole er gitt ved
\[\Phi = L \cdot I\]
Spenning
\[u(t) = L \cdot i\,'(t)\]
Strøm
\[i(t) = I_0+\frac{1}{L}\,\int_{t_0}^t u(x)\,dx\]
Energien i en spole er gitt ved
\[W = \frac{1}{2}\,L\,I^2\]

\subsection{Kondensator}
Ved $t = 0$ er det kontinuitet i kondensatorspenning. Ved $t \rightarrow \infty \implies i(t) \rightarrow 0$ (åpen sløyfe)\\\\
Spenningsdifferansen i en kondensator produserer et elektrisk felt.
Feltstyrken er gitt ved
\[\epsilon = \frac{U}{d}\]
der $d$ er avstanden.\\\\
Strøm
\[i(t) = C \cdot u\,'(t)\]
Spenning
\[u(t) = U_0+\frac{1}{C}\,\int_{t_0}^t i(x)\,dx\]
Energien i en kondensator er gitt ved
\[W = \frac{1}{2}\,C\,U^2\]

\subsubsection{Kapitans}
Er gitt ved
\[C = \frac{Q}{U}\]
Total kapitans i seriekobling er gitt ved
\begin{align*}
    C_{ser} &= \left( \sum_i \frac{1}{C_i} \right)^{-1} \\
    &= \left(\frac{1}{C_1} + \frac{1}{C_2} + ... + \frac{1}{C_n}\right)^{-1}
\end{align*}
Total kapitans i parallellkobling er gitt ved
\begin{align*}
    C_{par} &= \sum_i C_i \\
    &= C_1 + C_2 + ... + C_n
\end{align*}

\subsection{Kirchhoffs lover}
\subsubsection{Strømloven}
I et forgreningspunkt er summen av strømmer inn lik summen av strømmer ut. Alternativt, la $I_i$ angi strømmer inn til et forgreningspunkt, da gjelder
\[\sum_i I_i = 0\]
\subsubsection{Spenningsloven}
Summen av alle spenningsendringer i en sløyfe er lik 0. La $U_i$ angi spenninger over komponentene i en sløyfe, da gjelder
\[\sum_i U_i = 0\]

\subsection{Kildekonvertering}
\subsubsection{Thévenin-ekvivalent}
\begin{enumerate}
    \item Definer og marker utgangsterminalgrensesnittet (typisk mot lastmotstanden).
    \item Fjern alt utenfor terminalgrensesnittet.
    \item Regn ut spenningen mellom de åpne terminalene. Dette er $U_{Th}$.
    \item Kortslutt terminalene, og regn ut kortslutningsstrømmen $I_k$. $R_{Th}$ er gitt ved $\frac{U_{Th}}{I_k}$.
\end{enumerate}
Alternativt kan en finne $R_{Th}$ ved å nullstille alle kildene og regne ut resistansen mellom åpne terminaler.\\\\
Den ekvivalente Thévenin-kretsen er da bygget opp av en spenningskilde med $U_{Th}$ og en resistans $R_{Th}$ i serie.

\subsubsection{Norton-ekvivalent}
Finn $U_{Th}$ og $R_{Th}$ på samme måte som i en Thévenin-ekvivalent. \\\\
Den ekvivalente Norton-kretsen er da bygget opp av en strøm med $I = \frac{U_{Th}}{R_{Th}}$ og en resistans $R_{Th}$ i parallell.

\subsection{Avhengige spennings- og strømkilder}
Disse er kilder som ikke har fast verdi. Verdien er proposjonal med en styrestrøm eller styrespenning et annet sted i koblingen.
\begin{itemize}
    \item Spenningsforsterkning: $U_0 = A \cdot U_x$
    \item Strømforsterkning: $I_0 = B \cdot I_x$
    \item Transkonduktans: $I_0 = g \cdot U_x$
    \item Transresistans: $U_0 = r \cdot I_x$
\end{itemize}

\subsection{Knutespenningsmetoden}
En systematisk fremgangsmåte for å analysere en vilkårlig kobling. Metoden virker likevel ikke for grener med spenningskilde uten resistans, da anvender vi superknutemetoden.
\begin{enumerate}
    \item Finn og marker alle forgreningspunkter. Ett av disse skal være jordreferanse. Resten skal indekseres $a, b, c, ...$ . Jord er ofte negativ terminal på en kilde.
    \item Bruker Kirchhoffs 1. lov (strømloven) i hvert indekserte knutepunkt. Sett opp uttrykk for hver av grenstrømmene ved hjelp av spenningsreferansen mot forgreningspunktet i motsatt ende av grenen.
    \item Resultatet er $n$ knutelikninger der det finnes $n$ ulike knutespenninger. Disse ordnes i et likningssett og løses ved for eksempel innsetting, Gauss-eliminasjon, Cramers regel eller matriseinvertering.
    \item Grenstrømmene kan eventuelt finnes ved hjelp av Ohms lov og andre regler.
\end{enumerate}

\subsection{Superknutemetoden}
Superknutemetoden er en versjon av knutespenningsmetoden som brukes for grener med spenningskilde utens resistans. Metoden går ut på å definere en felles \textit{superknute} for forgreningspunktene på hver side av spenningskilden. Obs: dette gir flere ukjente enn likninger, men man bruker spenningsdiffereransen mellom forgreningspunktene som tilleggslikning. 

\subsection{Superposisjonsmetoden}
Superposisjonsmetoden kan \textbf{ikke} brukes dersom det er ulineære komponenter i koblingen.
\begin{enumerate}
    \item Velg ut en kilde, og nullstill alle de andre. \begin{itemize}
        \item Nullstille spenningskilde: $U = 0V$, dvs. kortslutning.
        \item Nullstille strømkilde: $I = 0A$, dvs åpen sløyfe / brudd.
    \end{itemize}
    Analyser denne koblingen med tanke på den aktuelle strømmen eller spenningen. Resultat: strømbidrag / spenningsbidrag.
    \item Repeter steg 1 for alle kilder etter tur.
    \item Delbidragene skal adderes for å finne den aktuelle strømmen / spenningen.
\end{enumerate}

\subsection{Effektiv verdi - RMS}
Effektiv verdi, eller Root Mean Square, er den konstante verdien som leverer samme effekt som et varierende DC-signal ville levert i den samme belastningen.\\\\
La $V(t)$ være et varierende signal med periode $T$. Da er effektiv verdi gitt ved
\[V_{RMS} = \sqrt{\frac{\int_0^T V(t)\,dt}{T}}\]
RMS for noen vanlige signaler
\begin{itemize}
    \item Sinussignal med amplitude A: $V_{EFF} = \frac{A}{\sqrt{2}}$
    \item Sagtannsignal med amplitude A: $V_{EFF} = \frac{A}{\sqrt{3}}$
    \item Firkantsignal med amplitude A: $V_{EFF} = A$
\end{itemize}

\subsection{PMOS- og NMOS-transistorer}
\subsection{Hvordan disse kan brukes som brytere}
Virkemåten er baser på at konduktiviteten (ledningsevnen) til en ledende kanal i 
halvlederen kan varieres ved at en ekstern spenning (gate-spenningen) varieres. 
Anrikningstype MOSFET er normalt av (ingen ledende kanal) og transistoren 
fungerer som en åpen bryter mellom Drain og Source.
Ved å påtrykke en negativ spenning på Gate i en PMOS-transistor opprettes en p-kanal mellom Drain og Source og transistoren fungerer da som en lukket bryter sett 
mellom Drain og Source. I en NMOS-transistor må det påtrykkes en positiv spenning 
på Gate for å opprette en n-kanal mellom Drain og Source.
At kanalene (P-kanal og N-kanal) opprettes skyldes felteffekt.
Transistoren styres vha Gate-spenningen. Det går ingen strøm inn i Gate pga det 
isolerende oksid-laget.
\subsection{Maksimal frekvens}
Det vil alltid være en viss motstand i n-kanalen og p-kanalen som genereres. Denne 
motstanden kan ekvivaleres med en R.
Mellom Gate og halvlederen etableres det en ladning på hver side av oksid-laget. 
Dette gir opphav til en kapasitans C. Å opprette ladning eller flytte ladning vil ta en viss tid gitt av størrelsen på R og C i 
vår enkle modell. Hastigheten på hvor fort transistorene kan slås av/på er gitt av 
tidskonstanten av type $\tau = RC$.
\subsection{CMOS-transistor og inverterer}
Kombinasjonen av PMOS- og NMOS-transistorer danner et komplementært sett av 
transistorer som kalles CMOS-transistorer (Complementary Metal Oxide 
Semiconductor). I dette tilfellet er disse transistorene brukt i en inverter.

\section{Digitalteknikk}

\subsection{Toerkomplement}
I et tall på binær toerkomplement-form teller det første bitet negativt.\\\\
Omgjøring mellom standard binær og toerkomplement (og tilbake!) gjøres ved
\[T = \overline{B} + 1\]
\begin{table}[h]
    \centering
    \begin{tabular}{c|c|c}
    Desimaltall & Binærtall & Binærtall \\
    \hline
     & Sign-magnitude & 2's komplement \\
    \hline
    7   & 0111          & 0111\\
    6   & 0110          & 0110\\
    5   & 0101          & 0101\\
    4   & 0100          & 0100\\
    3   & 0011          & 0011\\
    2   & 0010          & 0010\\
    1   & 0001          & 0001\\
    0   & 0000 / 1000   & 0000\\
    -1  & 1001          & 1111\\
    -2  & 1010          & 1110\\
    -3  & 1011          & 1101\\
    -4  & 1100          & 1100\\
    -5  & 1101          & 1011\\
    -6  & 1110          & 1010\\
    -7  & 1111          & 1001\\
    -8  & -             & 1000
    \end{tabular}
    \caption{Sign-magnitude og 2's komplement}
\end{table}


\subsection{Mintermer og makstermer}
En minterm $m_i$ har den egenskapen at den er lik 1 i nøyaktig én rad i sannhetstabellen. En maksterm $M_i$ er lik 0 i nøyaktig én rad i sannhetstabellen.\\\\
Eksempel:
\begin{itemize}
    \item $F = \overline{x}\overline{y}\overline{z} + \overline{x}\overline{y}z + xyz$
    \item $F = m_0 + m_1 + m_2 = \sum \, (0, 1, 7)$
    \item $F = M_2 \cdot M_3 \cdot M_4 \cdot M_5 \cdot M_6 = \prod \, (2, 3, 4, 5, 6)$
\end{itemize}

\subsection{Karnaughdiagram og kubediagram}

\subsection{Tabellmetoden}

\subsection{Overflyt}
Hvis man legger sammen to tall A og B med samme fortegn, men R får forskjellig fortegn, har man overflyt. Da er resultatet ugyldig.

\subsection{Oppsetningstid og holdetid}
Oppsetningstid er tiden signalet må være gyldig før klokkeendring. Holdetid er tiden signalet må være gyldig etter klokkeendring.

\subsection{Vipper (flipflops)}

\subsubsection{D-vippe}
$Q_{next} = D$.
\begin{table}[h]
    \centering
    \begin{tabular}{c|c|c}
    $Q$ & $Q_{next}$ & $D$ \\
    \hline
    0 & 0 & 0\\
    0 & 1 & 1\\
    1 & 0 & 0\\
    1 & 1 & 1
    \end{tabular}
    \caption{Eksitasjonstabell, D-vippe}
\end{table}

\subsubsection{T-vippe}
Høy $T$ inverterer.
\begin{table}[h]
    \centering
    \begin{tabular}{c|c|c}
    $Q$ & $Q_{next}$ & $T$ \\
    \hline
    0 & 0 & 0\\
    0 & 1 & 1\\
    1 & 0 & 1\\
    1 & 1 & 0
    \end{tabular}
    \caption{Eksitasjonstabell, T-vippe}
\end{table}

\subsubsection{SR-vippe}
Høy $S$, lav $R$ gir $1$. Høy $R$, lav $S$ gir $0$. 
\begin{table}[h]
    \centering
    \begin{tabular}{c|c|c|c}
    $Q$ & $Q_{next}$ & $S$ & $R$\\
    \hline
    0 & 0 & 0 & $X$\\
    0 & 1 & 1 & 0\\
    1 & 0 & 0 & 1\\
    1 & 1 & $X$ & 0
    \end{tabular}
    \caption{Eksitasjonstabell, SR-vippe}
\end{table}

\subsubsection{JK-vippe}
Lik SR-vippe, men høy høy gir toggle.
\begin{table}[h]
    \centering
    \begin{tabular}{c|c|c|c}
    $Q$ & $Q_{next}$ & $J$ & $K$\\
    \hline
    0 & 0 & 0 & $X$\\
    0 & 1 & 1 & $X$\\
    1 & 0 & $X$ & 1\\
    1 & 1 & $X$ & 0
    \end{tabular}
    \caption{Eksitasjonstabell, SR-vippe}
\end{table}

\subsection{Kritisk sti}
Kritisk sti er maksimal forsinkelse mellom inngang og utgang.

\subsection{Maksimal klokkefrekvens}
Maksimal klokkefrekvens er gitt ved
\[f_{max} = \frac{1}{t_{min}} \]



\end{document}