\documentclass{article}

% Language setting
% Replace `english' with e.g. `spanish' to change the document language
%\usepackage[english]{babel}

% Set page size and margins
% Replace `letterpaper' with`a4paper' for UK/EU standard size
\usepackage[letterpaper,top=2cm,bottom=2cm,left=3cm,right=3cm,marginparwidth=1.75cm]{geometry}

% Useful packages
\usepackage{amsmath}
\usepackage{graphicx}
\usepackage[colorlinks=true, allcolors=blue]{hyperref}

% Number sets
\usepackage{amsfonts}

\title{Krets og digitalteknikk}
\author{Magne Tenstad}

\begin{document}

\maketitle

\clearpage

\tableofcontents

\clearpage

\section{Krets}


\subsection{Terminologi og definisjoner}


\subsubsection{Terminologi}
\begin{table}[h]
    \centering
    \begin{tabular}{c|c|c}
                    & Symbol & Enhet \\
        \hline
        Ladning     & Q & C, coulomb \\
        Strøm       & I & A, ampere \\
        Resistans   & R & $\Omega$, ohm \\
        Spenning    & U & V, volt \\
        Energi      & W & J, joule \\
        Effekt      & P & W, watt \\
        Konduktans  & G & S, siemens
    \end{tabular}
\end{table}


\subsubsection{Passiv fortegnskonvensjon}
Gjennom en kilde går strømretningen inn på negativ terminal og ut fra positiv terminal. Gjennom en motstand går strømretningen inn på positiv terminal og ut fra negativ terminal.


\subsubsection{Ohms lov}
Proposjonalitet mellom spenning over og strøm gjennom en motstand.
\[U = R \cdot I\]
med $P = U \cdot I$ følger det at
\[P = R \cdot I^2\]
og at
\[P = \frac{U^2}{R}\]


\subsubsection{Konduktans}
Konduktans angir ledningsevne, og er gitt ved
\[G = \frac{1}{R}\]
Konduktans kan være praktisk der resistansen er og skal være lav.


\subsubsection{Effekt og passiv fortegnskonvensjon}
La $k$ være en komponent, $U$ være spenningen over $k$ og $I$ være strømmen gjennom $k$. Effekten forbrukt i komponenten er gitt ved
\begin{itemize}
    \item Hvis strømretningen oppfyller den passive fortegnskonvensjonen: $P = U \cdot I$
    \item Hvis strømretningen ikke oppfyller den passive fortegnskonvensjonen: $P = - U \cdot I$
\end{itemize}





\section{Digitalteknikk}





\end{document}