\documentclass{article}

% Language setting
% Replace `english' with e.g. `spanish' to change the document language
%\usepackage[english]{babel}

% Set page size and margins
% Replace `letterpaper' with`a4paper' for UK/EU standard size
\usepackage[letterpaper,top=2cm,bottom=2cm,left=3cm,right=3cm,marginparwidth=1.75cm]{geometry}

% Useful packages
\usepackage{amsmath}
\usepackage{graphicx}
\usepackage[colorlinks=true, allcolors=blue]{hyperref}

% Number sets
\usepackage{amsfonts}

\title{Matematikk 2 Kompendium}
\author{Magne Tenstad}

\begin{document}

\maketitle

\clearpage

\tableofcontents

\clearpage




\section{Ulike koordinatsystemer}


\subsection{Kartesiske koordinater}
Kartesiske koordinater er det vanligste koordinatsystemet. Koordinataksene $x$, $y$ og $z$ står vinkelrett på hverandre.


\subsection{Polarkoordinater}
Polarkoordinater er vanlige å bruke dersom en har å gjøre med sirkulære objekter i $\mathbb{R}^2$. $r$ bestemmer lengden fra origo, og $\theta$ bestemmer vinkelen. Det omgjøres fra kartesiske koordinater ved
\begin{gather*}
    x = r\cos{\theta}\\
    y = r\sin{\theta}\\
    dx\,dy = r\,dr\,d\theta
\end{gather*}
Vanlige grenser er
\begin{gather*}
    0 \leq r \leq R\\
    0 \leq \theta \leq 2\pi
\end{gather*}
der R er radien til sirkelen.


\subsection{Sylinderkoordinater}
Sylinderkoordinater er en utvidelse av polarkoordinater, som inkluderer en $z$-akse.
Det omgjøres fra kartesiske koordinater ved
\begin{gather*}
    x = r\cos{\theta}\\
    y = r\sin{\theta}\\
    z = z\\
    dx\,dy\,dz = r\,dr\,d\theta\,dz
\end{gather*}
Vanlige grenser er
\begin{gather*}
    0 \leq r \leq R\\
    0 \leq \theta \leq 2\pi\\
    0 \leq z \leq H
\end{gather*}
der R er radien til sylinderen og H er høyden.


\subsection{Kulekoordinater}
Kulekoordinater er vanlige å bruke dersom en har å gjøre med kuleformede objekter i $\mathbb{R}^3$. $\rho$ bestemmer lengden fra origo, $\theta$ bestemmer vinkelen i $xy$-planet og $\phi$ bestemmer vinkelen i $zr$-planet. Det omgjøres fra kartesiske koordinater ved
\begin{gather*}
    x = \rho\sin{\phi}\cos{\theta}\\
    y = \rho\sin{\phi}\sin{\theta}\\
    z = \rho\cos{\phi}\\
    dx\,dy\,dz = \rho^2\sin(\phi)\,d\rho\,d\theta\,d\phi
\end{gather*}
Vanlige grenser er
\begin{gather*}
    0 \leq \rho \leq R\\
    0 \leq \theta \leq 2\pi\\
    0 \leq \phi \leq \pi
\end{gather*}
der R er radien til kulen.


\subsection{Jacobi-determinanten}
Gitt en koordinattransformasjon
\[T(u, v) = (x(u, v),\,y(u, v))\]
så er Jacobi-determinanten til $T$ definert som
\begin{align*}
    J(u, v) = \frac{\partial (x, y)}{\partial (u, v)} =
    \begin{vmatrix}
        \frac{\partial x}{\partial u} & \frac{\partial x}{\partial v}\\\\
        \frac{\partial y}{\partial u} & \frac{\partial y}{\partial v}
    \end{vmatrix}
    = \frac{\partial x}{\partial u} \frac{\partial y}{\partial v} - \frac{\partial x}{\partial v} \frac{\partial y}{\partial u}
\end{align*}
og
\[dA = dx\,dy = \left| J(u, v) \right| du\,dv\]
Gitt en koordinattransformasjon
\[T(u, v, w) = (x(u, v, w),\,y(u, v, w),\,z(u, v, w))\]
så er Jacobi-determinanten til $T$ definert som
\begin{align*}
    J(u, v, w) = \frac{\partial (x, y, z)}{\partial (u, v, w)} =
    \begin{vmatrix}
        \frac{\partial x}{\partial u} & \frac{\partial x}{\partial v} & \frac{\partial x}{\partial w}\\\\
        \frac{\partial y}{\partial u} & \frac{\partial y}{\partial v} & \frac{\partial y}{\partial w}\\\\
        \frac{\partial z}{\partial u} & \frac{\partial z}{\partial v} & \frac{\partial z}{\partial w}
    \end{vmatrix}
\end{align*}
og
\[dV = dx\,dy\,dz = \left| J(u, v, w) \right| du\,dv\,dw\]

\subsubsection{Eksempel: Polarkoordinater}
La
\[T(r, \theta) = (r\cos{\theta}, r\sin{\theta})\]
da har vi at
\[J(r, \theta) = \frac{\partial (x, y)}{\partial (r, \theta)} =
    \begin{vmatrix}
        \frac{\partial (r\cos{\theta})}{\partial r} & \frac{\partial (r\cos{\theta})}{\partial \theta}\\\\
        \frac{\partial (r\sin{\theta})}{\partial r} & \frac{\partial (r\sin{\theta})}{\partial \theta}
    \end{vmatrix}
    =
    \begin{vmatrix}
        \cos{\theta} & -r\sin{\theta}\\
        \sin{\theta} & r\cos{\theta}
    \end{vmatrix}
    = r
    \]
og
\[dx\,dy = r\,dr\,d\theta\]

\subsubsection{Eksempel: Oppgave 5 Høst 2018}
Vi ønsker å gjøre transformasjonen $u = x^2 + y^2$, $v = \frac{y^2}{x}$. Merk at det her er $u$ og $v$ som er uttrykt ved $x$ og $y$. Dette gir inverstransformasjonen
\[T^{-1}(x, y) = (x^2+y^2, \frac{y^2}{x})\]
da har vi at
\[J'(x, y) = \frac{\partial (u, v)}{\partial (x, y)} =
    \begin{vmatrix}
        \frac{\partial (x^2 + y^2)}{\partial x} & \frac{\partial (x^2 + y^2)}{\partial y}\\\\
        \frac{\partial (\frac{y^2}{x})}{\partial x} & \frac{\partial (\frac{y^2}{x})}{\partial y}
    \end{vmatrix}
    =
    \begin{vmatrix}
        2x & 2y\\
        -\frac{y^2}{x^2} & \frac{2y}{x}
    \end{vmatrix}
    = 2y \left(2 + \frac{y^2}{x^2}\right)
    \]
der $J'$ er Jacobi-determinanten til inverstransformasjonen $T^{-1}$ Igjen, merk at vi har gjort transformasjonen "andre veien", så
\[du\,dv = |J'(x, y)|\,dx\,dy \,\implies\, dx\,dy = \frac{1}{2y \left(2 + \frac{y^2}{x^2}\right)}\,du\,dv\]



\clearpage




\section{Kurver}


\subsection{Buelengde}
La  $C$ være en glatt kurve parametrisert ved $r(t)$, $a\leq t \leq b$. Da er lengden av $C$  gitt ved
\[S = \int_a^b \left| r'(t) \right| dt\]


\subsection{Enhetstangent, enhetsnormal og krumning}
La $r(t)$ være en glatt parametrisering av en kurve $C$.
\begin{itemize}
    \item Enhetstangent: $\hat{T}(t) = \frac{r'(t)}{\left| r'(t) \right|}$
    \item Enhetsnormal: $\hat{N}(t) = \frac{\hat{T}'(t)}{\left| \hat{T}'(t) \right|}$
    \item Krumning: $K(t) = \frac{\left| \hat{T}'(t) \right|}{\left| r'(t) \right|}$
\end{itemize}
For normalvektor til et plan, se \ref{Tangentplan og normalvektor}

\subsection{Areal av omdreiningslegeme}
La $C$ være en kurve parametrisert ved $r(t) = (X(t),\,Y(t))$, $a \leq t \leq b$. \\\\
Overflatearealet $A_x$ som fremkommer ved å dreie $C$ om $x$-aksen er gitt ved
\[A = 2\pi \int_a^b \left| Y(t) \right| \left| r'(t) \right| dt \]
Overflatearealet $A_y$ som fremkommer ved å dreie $C$ om $y$-aksen er gitt ved
\[A = 2\pi \int_a^b \left| X(t) \right| \left| r'(t) \right| dt \]


\subsection{Areal omsluttet av parametrisk kurve}
La $C$ være en kurve parametrisert ved $r(t) = (X(t), Y(t))$, $a \leq t \leq b$, og anta at $X(t)$ er deriverbar og $Y(t)$ er kontinuerlig på $\left[ a, b \right]$. La $A$ være arealet begrenset av $C$, $x$-aksen og linjene $x=X(a)$ og $x=X(b)$. Da gjelder
\begin{itemize}
    \item Hvis $X' \cdot Y \geq 0$ på $\left[ a, b \right]$, så er $A = \int_a^b X'(t)\,Y(t)\,dt$
    \item Hvis $X' \cdot Y \leq 0$ på $\left[ a, b \right]$, så er $A = -\int_a^b X'(t)\,Y(t)\,dt$
\end{itemize}


\clearpage


\section{Funksjoner av flere variable}


\subsection{Gradient}
Gradienten til et skalarfelt er et vektorfelt der vektoren i et hvert punkt peker i retningen til den største økningen i skalarfeltet. Lengden av vektoren er et uttrykk for endringen til skalarfeltet i retning av vektoren.\\\\
La $F$ være et skalarfelt i $\mathbb{R}^2$. Da er $\nabla F$ gitt ved
\[\nabla F(x,y) = \left(\frac{\partial F}{\partial x},\,\frac{\partial F}{\partial y} \right)\]
La $F$ være et skalarfelt i $\mathbb{R}^3$. Da er $\nabla F$ gitt ved
\[\nabla F(x,y,z) = \left(\frac{\partial F}{\partial x},\,\frac{\partial F}{\partial y},\,\frac{\partial F}{\partial z} \right)\]


\subsection{Retningsderivert}
La $f$ være deriverbar i  $a \in D_f$ og la u være en retningsvektor. Da den retningsderiverte gitt ved
\[D_{u} f(a) = \nabla f(a) \cdot u\]

\subsubsection{Egenskaper ved den retningsderiverte}
Husk at prikkproduktet mellom to vektorer kan uttrykkes ved vinkelen $\alpha$ mellom dem:
\[D_{u} f(a) = \nabla f(a) \cdot u = \left| \nabla f(a) \right| \left| u \right| \cos{\alpha} = \left| \nabla f(a) \right| \cos{\alpha}\]
Dette gir følgende egenskaper:
\begin{itemize}
    \item $f$ øker mest når $\alpha = 0$, altså når $u = \frac{\nabla f(a)}{\left| \nabla f(a) \right|}$.
    \item $f$ minker mest når $\alpha = \pi$, altså når $u = - \frac{\nabla f(a)}{\left| \nabla f(a) \right|}$.
    \item $f$ endres ikke når $\alpha = \pi / 2$, altså når $u \cdot \nabla f(a) = 0$.
\end{itemize}


\subsection{Tangentplan og normalvektor} \label{Tangentplan og normalvektor}
Likningen for tangentplanet til $z = f(x, y)$ i punktet $(a, b, f(a, b))$ er gitt ved
\[z = f(a, b) + \frac{\partial f}{\partial x}(a, b)(x - a) + \frac{\partial f}{\partial y}(a, b)(y - b)\]
Normalen til $z = f(x, y)$ i punktet $(a, b)$ er gitt ved
\[\hat{N} = \left( \frac{\partial f}{\partial x}(a, b),\,\frac{\partial f}{\partial y}(a, b),\,-1 \right)\]
Gitt tre punkter $A$, $B$ og $C$, er normalen til planet som dannes av punktene gitt ved kryssproduktet av to vektorer mellom punktene
\[\hat{N} = \overrightarrow{AB} \times \overrightarrow{AC}\]

\subsection{Å finne og klassifisere punkter}

\subsubsection{Kritiske punkter}
Kritiske punkter for $f$ er punkter der
\begin{itemize}
    \item $\nabla f = \overrightarrow{0}$
    \item Singulære punkter: der $\nabla f$ er udefinert
\end{itemize}
Merk! Dersom en skal finne ekstremalpunkter på et lukket område, må en også undersøke randpunktene. Dette gjøres ved parametrisering av randen eller ved Lagranges multiplikatormetode med randen som betingelse.\\\\
Indre kritiske punkter (ikke randpunkter) kan klassifiseres ved andrederiverttesten.

\subsubsection{Lagranges multiplikatormetode}
En metode for bestemmelse av ekstremalverdier av en funksjon med flere variabler når disse må oppfylle en eller flere bibetingelser.\\\\
Eksempel: Gitt $f(x, y) = 4x + y$, med betingelse $g(x, y) = x^2 + y^2 - 4 = 0$
\begin{enumerate}
    \item La Lagrangefunksjonen $L(x, y, \lambda) = f(x, y) + \lambda g(x, y) = 4x + y + \lambda x^2 + \lambda y^2 - 4 \lambda$
    \item Finn løsninger av likningssettet $\nabla L = \overrightarrow{0}$
\end{enumerate}
Løsningene vil være ekstremalpunkter for $f$. Gitt flere betingelser legger man dem til i $L$, med hver sin variabel.

\subsubsection{Andrederiverttesten}
La $f(x, y)$ være en funksjon av to variabler. Anta at $(a, b)$ er et indre kritisk punkt og at de andrederiverte av f er kontinuerlige i nærheten av $(a, b)$. Sett
\begin{align*}
    A &= \frac{\partial^2 f}{\partial x^2} (a, b),\\
    B &= \frac{\partial^2 f}{\partial x\,\partial y} (a, b) = \frac{\partial^2 f}{\partial x\,\partial y} (a, b)\\
    C &= \frac{\partial^2 f}{\partial y^2} (a, b),
\end{align*}
og
\[D = AC-B^2\]
Da gjelder:
\begin{enumerate}
    \item Hvis $D < 0$, så er $(a, b)$ et sadelpunkt.
    \item Hvis $D > 0$ og $A > 0$, så er $(a, b)$ et lokalt minimumspunkt.
    \item Hvis $D > 0$ og $A < 0$, så er $(a, b)$ et lokalt maksimumspunkt.
    \item Hvis $D = 0$ gir testen ingen konklusjon.
\end{enumerate}


\subsection{Linjeintegral}
Linjeintegralet av et skalarfelt $f = f(x, y, z)$ langs en kurve $C \subset \mathbb{R}^3$ gitt ved en glatt parametrisering $r(t)$, $t \in [a,b]$ er gitt ved
\[\int_c f\,ds = \int_a^b f(r(t))\,\left| r'(t) \right| dt\]
\subsubsection{Lengde av kurve}
Det følger at lengden av en kurve $C \subset \mathbb{R}^3$ gitt ved en glatt parametrisering $r(t)$, $t \in [a,b]$ er gitt ved
\[\int_c 1\,ds = \int_a^b \left| r'(t) \right| dt\]
\subsubsection{Masse av streng}
Dersom $\delta$ angir massetetthet per lengdeenhet, og kurven $C \subset \mathbb{R}^3$ gitt ved en glatt parametrisering $r(t)$, $t \in [a,b]$ angir posisjonen til en streng i rommet, er massen til strengen gitt ved
\[m = \int_c \delta\,ds = \int_a^b \delta (r(t))\,\left| r'(t) \right| dt\]

\subsection{Flateintegral}
Flateintegralet av et skalarfelt $f = f(x, y, z)$ over en flate $S \subset \mathbb{R}^3$ gitt ved en parametrisering $r(u, v)$, $u_0 \leq u \leq u_1$, $v_0 \leq v \leq v_1$ er gitt ved
\[\iint_S f\,dS=\int_{v_0}^{v_1}\int_{u_0}^{u_1} f(r(u,v)) \left| \frac{\partial r}{\partial u} \times \frac{\partial r}{\partial v} \right|\,du\,dv\]
der
\begin{align*}
    \frac{\partial r}{\partial u} &= \left( \frac{\partial x(u, v)}{\partial u},\, \frac{\partial y(u, v)}{\partial u},\, \frac{\partial z(u, v)}{\partial u} \right)\\
    \frac{\partial r}{\partial v} &= \left( \frac{\partial x(u, v)}{\partial v},\, \frac{\partial y(u, v)}{\partial v},\, \frac{\partial z(u, v)}{\partial v} \right)
\end{align*}
\subsubsection{Areal av flate}
Det følger at arealet av en flate $S \subset \mathbb{R}^3$ gitt ved en parametrisering $r(x, y) = (x, y, f(x, y))$, $x_0 \leq x \leq x_1$, $y_0 \leq y \leq y_1$ er gitt ved
\begin{align*}
    \iint_S 1\,dS &= \int_{y_0}^{y_1}\int_{x_0}^{x_1} \left| \frac{\partial r}{\partial x} \times \frac{\partial r}{\partial y} \right|\,dx\,dy\\
    &= \int_{y_0}^{y_1}\int_{x_0}^{x_1} \sqrt{1 + \left(\frac{\partial f}{\partial x}\right)^2 + \left(\frac{\partial f}{\partial y}\right)^2}\,dx\,dy
\end{align*}
\subsubsection{Masse av flate}
Dersom $\delta = \delta(x, y, z)$ angir massetetthet per lengdeenhet, og flaten $S \subset \mathbb{R}^3$ er gitt ved en parametrisering $r(x, y) = (x, y, f(x, y))$, $x_0 \leq x \leq x_1$, $y_0 \leq y \leq y_1$, er massen av flaten gitt ved
\begin{align*}
    \iint_S \delta\,dS &= \int_{y_0}^{y_1}\int_{x_0}^{x_1} \delta(x, y, f(x, y))\, \left| \frac{\partial r}{\partial x} \times \frac{\partial r}{\partial y} \right|\,dx\,dy\\
    &= \int_{y_0}^{y_1}\int_{x_0}^{x_1} \delta(x, y, f(x, y))\, \sqrt{1 + \left(\frac{\partial f}{\partial x}\right)^2 + \left(\frac{\partial f}{\partial y}\right)^2}\,dx\,dy
\end{align*}

\subsection{Middelverdisetningen}
Anta at $f(x, y)$ er kontinuerlig på et lukket, begrenset og sammenhengende område $D \in \mathbb{R}^2$. Da finnes det et punkt $(x_0, y_0) \in D$ der $f$ tar sin middelverdi (gjennomsnittsverdi), gitt ved
\[\overline{f} = f(x_0, y_0) = \frac{1}{areal(D)}\,\iint_D f(x, y)\,dA\]


\subsection{Massesenter}
Gitt en masse $m = f(\delta)$, er massesenteret gitt ved
\[(\overline{x}, \overline{y}, \overline{z}) = \left( \frac{f(x\delta)}{m}, \frac{f(y\delta)}{m}, \frac{f(z\delta)}{m} \right)\]


\clearpage




\section{Vektorfelt}


\subsection{Divergens}
Divergens gir størrelsen til en kilde eller et sluk i et gitt punkt i et vektorfelt, i form av en skalar med fortegn.\\\\
La $F(x, y) = \left(f_1(x, y), f_2(x, y)\right)$. Da er divergensen gitt ved
\[\nabla \cdot F = \frac{\partial f_1}{\partial x} + \frac{\partial f_2}{\partial y}\]
La $F(x, y, z) = \left(f_1(x, y, z), f_2(x, y, z), f_3(x, y, z)\right)$. Da er divergensen gitt ved
\[\nabla \cdot F = \frac{\partial f_1}{\partial x} + \frac{\partial f_2}{\partial y} + \frac{\partial f_3}{\partial z}\]

\subsubsection{Divergensteoremet} \label{Divergensteoremet}
La $D$ være et regulært område i $\mathbb{R}^3$ med rand $S = \partial D$ som er en orientert og lukket flate, der enhetsnormalen $\hat{N}$ peker ut av $D$. Hvis $F$ er et glatt vektorfelt definert på $D$, så er
\[\iiint_D \nabla \cdot F\,dV = \iint_S F \cdot \hat{N}\,dS\]

\subsubsection{Divergensteoremet i planet}
La $R$ være et regulært område i $xy$-planet. Anta at randen $C = \partial R$ består av en eller flere glatte, enkle, lukkede kurver som er positivt orienterte med hensyn på $R$. La $\hat{N}$ være den ytre enhetsnormalen på $C$. Hvis $F(x, y)=(f_1(x, y), f_2(x, y))$ er et glatt vektorfelt definert på $R$, så er
\[\iint_R \nabla \cdot F\,dA = \oint_C F \cdot \hat{N}\,ds\]


\subsection{Curl}
Curl beskriver den infinitesimale rotasjonen av et vektorfelt, i form av en vektor i ethvert punkt i feltet. Egenskapene til denne vektoren (lengden og retningen) beskriver rotasjonen i punktet.\\\\
La $F(x, y) = \left(f_1(x, y), f_2(x, y)\right)$. Da er curlen gitt ved
\[\nabla \times F = \left(0, 0, \frac{\partial f_2}{\partial x} - \frac{\partial f_1}{\partial y} \right)\]
La $F(x, y, z) = \left(f_1(x, y, z), f_2(x, y, z), f_3(x, y, z)\right)$. Da er curlen gitt ved
\[\nabla \times F = \left( \frac{\partial f_3}{\partial y} - \frac{\partial f_2}{\partial z}, \frac{\partial f_1}{\partial z} - \frac{\partial f_3}{\partial x}, \frac{\partial f_2}{\partial x} - \frac{\partial f_1}{\partial y} \right)\]

\subsubsection{Stokes' teorem} \label{Stokes' teorem}
La $S$ være en stykkevis glatt, orientert flate i $\mathbb{R}^3$ med enhetsnormal $\hat{N}$, der randen til $S$, $C = \partial S$, består av en eller flere stykkevis glatte, lukkede kurver med orientering bestemt av orienteringen til $S$. Hvis $F$ er glatt på en åpen mengde som inneholder $S$, så er
\[\oint_C F \cdot dr = \iint_S (\nabla \times F) \cdot \hat{N}\,dS\]

\subsubsection{Greens Teorem (Stokes' teorem i planet)} \label{Greens Teorem}
La $S$ være et regulært område i $xy$-planet. Anta at randen $C$ består av en eller flere stykkevis glatte, enkle, lukkede kurver med positiv orientering med hensyn på $R$. Hvis $F(x, y) = ((f_1(x, y), f_2(x, y))$ er et glatt vektorfelt definert på $R$, så er
\begin{align*}
    \iint_R \left( \frac{\partial f_2}{\partial x} - \frac{\partial f_1}{\partial y} \right) &= \oint_C f_1(x, y)\,dx + f_2(x, y)\,dy\\
    &= \oint_C F \cdot dr
\end{align*}


\subsection{Konservative vektorfelt}
Et vektorfelt $F$ i et område $D \subseteq \mathbb{R}^3$ er konservativt hvis og bare hvis det finnes et skalarfelt $\phi$ slik at
\[F(x,y,z) = \nabla \phi(x, y, z)\]
for alle $(x, y, z) \in D$. Vi kaller $\phi$ en potensialfunksjon for $F$ i $D$.\\\\
Det er verdt å merke seg at for et konservativt vektorfelt $F$ så er $\nabla \times F = \overrightarrow{0}$\\\\
For et konservativt vektorfelt $F$ med en potensialfunksjon $\phi$, og en kurve $C$ parameterisert ved $r(t)$, $a \leq t \leq b$ gjelder
\[\int_C F \cdot dr = \phi(b) - \phi(a)\]
dette henger sammen ved at for konservative vektorfelt er integralet uavhengig av veivalget \textit{mellom} start- og endepunkt av kurven.

\subsection{Linjeintegral i vektorfelt}
Linjeintegralet av et vektorfelt $F$ langs en kurve $C \subset \mathbb{R}^3$ gitt ved en glatt parametrisering $r(t)$, $t \in [a, b]$ er gitt ved
\[\int_C F \cdot dr = \int_a^b F(r(t)) \cdot \frac{\partial r}{\partial t}\,dt\]
Om C er lukket (eller "kan lukkes") kan Stokes' Teorem \ref{Stokes' teorem} eller Greens Teorem \ref{Greens Teorem} være nyttig.\\\\
Alternativt kan linjeintegralet uttrykkes ved enhetstangenten til $r$, $\hat{T}$
\[\int_C F \cdot dr = \int_C F \cdot \hat{T}\,ds\]
\subsection{Flateintegral i vektorfelt}
For et vektorfelt $F$ er fluksen gjennom en orientert flate $S$ gitt ved
\[\iint_S F \cdot \hat{N}\,dS\]
der $\hat{N}$ er enhetsnormalen til $S$ i positiv retning i forhold til orienteringen av $S$.\\\\
Om S er lukket (eller "kan lukkes") kan Divergensteoremet \ref{Divergensteoremet} være nyttig.


\clearpage




\end{document}
